\section{Installation des serveurs}
	\subsection{Hyperviseur}
		% Décrire le choix d'hyperviseur et sa configuration si besoin
Etant donné la configuration matérielle disponible, nous avons décidé d’installer un hyperviseur de type 1 sur nos serveurs physiques. Cette solution est la plus performante pour notre environnement car l’hyperviseur est directement au niveau de la couche matériel. Nous avons choisi d’utiliser VMware ESXi car c’est un hyperviseur de type 1 rapide, fiable et performant. Il dispose également d’un logiciel de gestion vSphere pour utiliser l’hyperviseur avec une interface graphique. De plus, par défaut, VMware permet d’utiliser les VM en mode pont avec un switch virtuel pour être relié directement au réseau local.

	\subsection{Machines virtuelles}
		% Décrire installation des machines virtuelles
Il faut créer une machine virtuelle en spécifiant les options voulues comme le disque dur, la mémoire vive, les processeurs, la carte réseau. Ensuite, il faut télécharger l’ISO d’installation sur le serveur pour que celle-ci soit utilisable par la VM. Puis, il faut insérer l’ISO dans la VM et lancer la VM, l’installation du système d’exploitation se fait de la même façon qu’avec un PC.

	\subsection{Communication inter-VMs}
		% Décrire notre architecture pour communiquer, commutateur, NAT, routeur...

Pour la communication entre les VM, nous avons mis en place un réseau physique entre deux serveurs éxécutant l'hyperviseur de type 1.

Ce réseau est un réseau local standard composé de deux serveurs, un switch et des postes de chaque technicien d'infrastructure.
ESXi dispose d'un système de switch virtuel permettant de former un pont entre les machines virtuelles et le réseau physique.
Ainsi, chaque machine virtuelle est directement connectée au réseau.

Pour le bon fonctionnement ce de réseau, un DHCP est installé sur le serveur de base du groupe distribuant les adresses IP et ayant enregistré les adresses statiques des serveurs.

Pour disposer d'un acces à internet, nous avons mis en place un routeur NAT sur une machine Arch Linux a l'aide d'IP tables.
Il suffit de configurer le système pour qu'il autorise le transfert de paquet puis configurer le par-feu interne du système en ajoutant les règles suivantes permettant de transmettres les packets.
L'adresse de la machine NAT est alors donnée dans le DHCP pour que chaque machine aie acces au réseau externe.

\begin{lstlisting}[caption=Commandes iptables permettant l'installation d'un NAT]
iptables -t nat -A POSTROUTING -o interface-externe -j MASQUERADE
iptables -A FORWARD -i interface-interne -o interface-externe -m state --state RELATED,ESTABLISHED -j ACCEPT
iptables -A FORWARD -i interface-interne -o interface-externe -j ACCEPT
\end{lstlisting}
