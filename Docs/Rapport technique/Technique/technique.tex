% Pas trop d'idée pour ce nom de section à revoir...
\section{Réponse au besoin}
	\subsection{Scripts}
	\subsection{GPO}
		\subsubsection{Partage Groupe}
			\paragraph{}
				Ce partage est disponible pour tous les services du groupe iSEC et se trouve sur le contrôleur de domaine principal. Il est ajouté comme lecteur réseau sur le poste de travail avec la lettre \texttt{G:} comme \textit{Groupe}. Pour créer ce partage sur le contrôleur de domaine maitre, il faut suivre la procédure suivante :

				\begin{enumerate}
					\item Créer le dossier qui sera partagé : sur le contrôleur de domaine ce dossier se trouve dans \texttt{C:\\Share\\} sous le nom de \texttt{Groupe}, comme tous les dossiers partagés.
					\item Il faut ensuite partager le dossier : 
						\begin{enumerate}
							\item Cliquer droit sur le dossier partagé
							\item Cliquer sur propriétés
							\item Dans l'onglet
						\end{enumerate}  
				\end{enumerate}
		\subsubsection{Partage Telecom}
			\paragraph{}
				Le partage Telecom se fait de la même façon
		\subsubsection{Partage par service}
		\subsubsection{Imprimantes}
		\subsubsection{Sécurité mot de passe}
		\subsubsection{Exécution automatique}
		\subsubsection{Fond d'écran}
		\subsubsection{Logiciel 7Zip}
		\subsubsection{...}